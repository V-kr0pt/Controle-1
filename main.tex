\documentclass{article}
\usepackage[portuges]{babel}
\usepackage{epsfig,subfigure,cite,graphicx}
\usepackage{indentfirst}
\setlength{\parindent}{20pt}
\usepackage[cmex10]{amsmath}
 \usepackage[left=2.5cm,top=2.5cm,right=2.5cm,bottom=2.5cm]{geometry}

% load package with ``framed'' and ``numbered'' option.
\usepackage[framed,numbered,autolinebreaks,useliterate]{mcode}

% something NOT relevant to the usage of the package.
\usepackage{url}
\setlength{\parindent}{0pt}
\setlength{\parskip}{18pt}
\title{CONTROLE 1}

% //////////////////////////////////////////////////

\begin{document}
\section*{ATIVIDADE 1 - CONTROLE 1}

\indent Aluno: Vitor de Sousa França\\
\indent Matrícula: 20180041455\\
\indent 22 de Outubro de 2021\\

\section*{Questão 1: Considerando-se o sistema:  }
    \begin{equation*}
        G = \frac{0.25(K_d s^2 + K_p s +K_i)}{s(s+1)(s+5)}
    \end{equation*}
    
    Em que $K_d = 150.88$, $K_p = 1373.92$ e $K_i = 5000$
    
    \subsection*{a) Utilizando MATLAB realizar as simulações do sistema em malha fechada no domínio do tempo contínuo.}

O sinal x(kT), com a frequência de amostragem desejada pode ser construido utilizando o Matlab através do script a seguir.

\begin{lstlisting}[language=Matlab,caption='Primeira Simulação']
%% Q1 a)
fo = 10; %freq principal do sinal
fs = 1000; %freq. de amostragem

Ts = 1/fs; %periodo de amostragem
T = 2*(fs/fo); %periodo de duas oscilacoes completas

k = 0:(T-1); %vetor contendo qtd de elementos = qtd de amostras
t = Ts*k; %distanciando igualmente as amostras de um periodo de amostragem

%criando o vetor sinal amostradocolor
x = cos(2*pi*fo*t) + 0.5*sin(2*pi*5*fo*t+pi/4);
\end{lstlisting}



\end{document}
